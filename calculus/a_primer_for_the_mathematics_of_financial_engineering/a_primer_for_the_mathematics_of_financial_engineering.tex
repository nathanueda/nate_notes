\documentclass[11pt]{article}
\usepackage{hyperref} 
\usepackage{amsmath, amsfonts, amssymb}
\usepackage{graphicx}
\usepackage{float}
\usepackage[margin=1in]{geometry}

\parindent0px

\emergencystretch=0pt
\pretolerance=150
\tolerance=10000
\hbadness=10000
\hfuzz=0pt

\title{A Primer for the Mathematics of Financial Engineering Notes}
\author{Nathan Ueda}
\date{\today} 

\begin{document}
\maketitle 
\pagebreak
\tableofcontents 
\pagebreak


\section{Calculus Review, Options, Put-Call Parity}
\subsection{Brief Review of Differentiation}

\textbf{Differentiation Basics}
\begin{itemize}
    \item Differentiation is a method used to compute the rate of change of a function $f(x)$
    with respect to its input $x$. This rate of change is known as the derivative of $f$ with
    respect to $x$. 
    \item The first derivative of a function $y=f(x)$ is denoted $\frac{dy}{dx}$, where $dy$
    denotes an infinitesimally small change in $y$ and $dx$ denotes an infinitesimally small 
    change in $x$. It is defined by 
    \[\frac{dy}{dx} = \lim_{h \to 0} \left[\frac{f(x+h) - f(x)}{h}\right]\]
    where as $h$ gets closer to 0, it approaches the slope at the tangent line
    \item The process of finding the derivative by taking this limit is known as 
    differentiation from first principles. In practice, it is not convenient to use this method.
    \item The derivatives of many functions can instead be found using standard derivatives in 
    conjunction with rules such as the chain rule, product rule, and quotient rule.
    \item Leibniz's Notation
    \begin{itemize}
        \item Given a function $y=f(x)$, the first derivative of $y$ with respect to $x$ is 
        denoted by 
        \[ \frac{dy}{dx} \]
        \item The second derivative of $y$ with respect to $x$ is denoted by 
        \[ \frac{d}{dx}\left[\frac{dy}{dx}\right]=\frac{d^2y}{dx^2}\]
        \item The $n$th derivative of $y$ with respect to $x$ is denoted by
        \[ \frac{d^ny}{dx^n}\]
        \item The derivative of $y$ with respect to $x$ at the point $x=a$ is denoted by 
        \[\frac{dy}{dx}|_{x=a}\]
    \end{itemize}
    \item The function $f(x)$ is differentiable if it is differentiable at all points $x$.
    \item 
\end{itemize}

\textbf{Product Rule}
\begin{itemize}
    \item If the two functions $f(x)$ and $g(x)$ are differentiable (i.e. their derivatives
    exist) then the product is differentiable and 
    \[(fg)' = f'g + fg'\]
\end{itemize}

\textbf{Quotient Rule}
\begin{itemize}
    \item If the two functions $f(x)$ and $g(x)$ are differentiable (i.e. their derivatives
    exist) then the quotient is differentiable and 
    \[\left(\frac{f}{g}\right)' = \frac{gf' - fg'}{g^2}\]
    \item This has the mnemonic
    \[\frac{\text{lo de hi} - \text{hi de lo}}{\text{lo lo}}\]
\end{itemize}

\textbf{Chain Rule}
\begin{itemize}
    \item  If the two functions $f(x)$ and $g(x)$ are differentiable then the composite
    function $f \circ g = f(g(x)) = F'(x)$ is differentiable and 
    \[ F'(x) = f'(g(x))g'(x)\]
    \item It is often used for power functions, exponential functions, and logarithmic 
    functions.
\end{itemize}

\subsection{Brief Review of Integration}

\textbf{Integration Basics}
\begin{itemize}
    \item Integration is a way to figure out what function we differentiated to get the 
    function $f(x)$.
\end{itemize}

\textbf{Indefinite Integrals}
\begin{itemize}
    \item Given a function $f(x)$, an antiderivative of $f(x)$ is any function $F(x)$ such 
    that $F'(x) = f(x)$. 
    \item If $F(x)$ is any antiderivative of $f(x)$ then the most general antiderivative of 
    $f(x)$ is called an indefinite integral and is denoted 
    \[\int f(x) dx = F(x) + c, \ \ \ c\text{ is an arbitrary constant}\]
    where $\int$ is the integral symbol, $f(x)$ is the integrand, $x$ is the integration 
    variable, and $c$ is the constant of integration. 
\end{itemize}

\textbf{Indefinite Integrals}
\begin{itemize}
    \item Suppose $f(x)$ is a continuous function on $[a,b]$ and also suppose that $F(x)$ is
    any antiderivative for $f(x)$, then 
    \[ \int_{a}^{b} f(x) dx = F(x)|_a^b = F(b) - F(a) \]
\end{itemize}

\textbf{Integration By Parts}
\begin{itemize}
    \item Easier formula
    \[ \int u \,dv = uv - \int v \,du\]
    \item To use this formula, we will need to identify $u$ and $dv$ and then use the formula.
    \item Choose $u$ such that
    \begin{itemize}
        \item The integral of $v$ is possible.
        \item The derivative of $u$ is better (reduced power)
    \end{itemize}
    \item Choosing $u$ order of operations
    \begin{itemize}
        \item L: Log functions
        \item I: Inverse trig functions
        \item P: Polynomials
        \item E: Exponentials
        \item T: Trig functions
    \end{itemize}
    \item It will not always be obvious what the correct values to choose are and, on 
    occassion, the wrong choice will be made. If this happens, we can always just go back and
    try a different set of choices. 
    \item Integration by parts is the counterpart for integration of the product rule.
\end{itemize}

\textbf{Integration By Substitution}
\begin{itemize}
    \item Integration by substituion formula
    \[ \int f(g(x)) g'(x) \,dx = \int f(u) \,du \text{\ \ \ where } u = g(x) \]
    \item Integration by substitution is the couterpart for integration of the chain rule.
\end{itemize}

\subsection{Differentiating Definite Integrals}

\subsection{Limits}

\subsection{L'Hopitals Rule and Connections to Taylor Expansions}

\subsection{Multivariable Functions}

\textbf{Partial Derivatives}
\begin{itemize}
    \item In practice, the partial derivative $\frac{\partial f}{\partial x_i} (x)$ is computed
    by considering the variables $x_1, \ldots, x_{i-1}, x_{i+1}, \ldots, x_n$ to be fixed, and
    differentiating $f(x)$ as a function of one variable $x_i$.
    \item Partial derivatives of higher order are defined similarly. The second order partial 
    derivative of $f(x)$ with respect to $x_i$ and then with respect to $x_j$, with $j \ne i$
    is defined as
    \[ \frac{\partial^2f}{\partial x_j \partial x_i} (x) = \frac{\partial}{\partial x_j} \left(
        \frac{\partial f}{\partial x_i}(x)\right) \]
\end{itemize}

\textbf{Gradient}
\begin{itemize}
    \item The gradient is a vector that points in the direction of maximum increase.
    \item An $n \times 1$ vectors that stores all the first order partial derivatives.
    \item Let $f : \mathbb{R}^n \to \mathbb{R}$ be a function of $n$ variables and assume that 
    $f(x)$ is differentiable with respect to all variables $x_i, \ i = 1, \ldots, n$. The 
    gradient $\nabla f$ of the function $f(x)$ is the following vector of $n$ components
    \[ 
    \nabla f(x) =  
    \begin{bmatrix}
        \frac{\partial f}{\partial x_1}(x) \\
        \vdots \\
        \frac{\partial f}{\partial x_n}(x) \\
    \end{bmatrix}
    \]
\end{itemize}

\textbf{Hessian}
\begin{itemize}
    \item An $n \times n$ matrix that stores all the second order partial derivatives.
    \item Let $f : \mathbb{R}^n \to \mathbb{R}$ be a function of $n$ variables. The Hessian of 
    $f(x)$ is the following matrix of $n \times n$ components
    \[
    H = \begin{bmatrix}
        \frac{\partial^2 f}{\partial x_1^2}(x) & \frac{\partial^2 f}{\partial x_2 \partial x_1}(x) & \cdots & \frac{\partial^2 f}{\partial x_n \partial x_1}(x) \\
        \frac{\partial^2 f}{\partial x_1 \partial x_2}(x) & \frac{\partial^2 f}{\partial x_2^2}(x) & \cdots & \frac{\partial^2 f}{\partial x_n \partial x_2}(x) \\
        \vdots & \vdots & \ddots & \vdots \\
        \frac{\partial^2 f}{\partial x_1 \partial x_n}(x) & \frac{\partial^2 f}{\partial x_2 \partial x_n}(x) & \cdots & \frac{\partial^2 f}{\partial x_n^2}(x) \\
    \end{bmatrix}    
    \]
\end{itemize}

\end{document}