\documentclass[11pt]{article}
\usepackage{hyperref} 
\usepackage{amsmath, amssymb, amsfonts}
\parindent0px

\title{SQL Notes}
\author{Nathan Ueda}
\date{\today} 

\begin{document}
\maketitle 
\pagebreak
\tableofcontents
\pagebreak

\section{Database Basics}
\subsection{Databases}
Database: A collection of data stored in a format that can easily be accessed.

\subsection{Database Management Systems}
Database Management System (DBMS): Software used to manage databases. It receives instructions
and executes them, sending results back. \\

Two categories of DBMS
\begin{enumerate}
    \item Relational: Data is stored in tables that are linked together by relationships. SQL
    is the language used to work with relational DBMS.
    \item Non-Relational (noSQL): No tables or relationships. NoSQL systems don't understand
    SQL (they have they own language).
\end{enumerate}

\subsection{Relational Database Management Systems}
Relational Database Management System (RDBMS): Software used to manage relational databases. 

Popular RDBMS
\begin{enumerate}
    \item MySQL
    \item SQL Server
    \item Oracle
\end{enumerate}

While each RDBMS is very similar to each other, they do have differences. 

\end{document}