\documentclass[11pt]{article}
\usepackage{hyperref} 
\usepackage{amsmath, amsfonts, amssymb}
\usepackage{graphicx}
\usepackage{float}
\usepackage[margin=1in]{geometry}

\parindent0px

\emergencystretch=0pt
\pretolerance=150
\tolerance=10000
\hbadness=10000
\hfuzz=0pt

\title{Probability and Statistics Notes}
\author{Nathan Ueda}
\date{\today} 

\begin{document}
\maketitle 
\pagebreak
\tableofcontents 
\pagebreak

\section{Introduction to Probability}
\subsection{The History of Probability}
\subsection{Interpretations of Probability}

Probability Interpretations
\begin{itemize}
    \item Frequency: If an experiment is carried out many times, the frequency with which a
    particular outcome occurred would define its probability.
    \item Classical: If an outcome of some experiment must be one of $n$ different, equally
    likely outcomes, the probability of each outcome is $\frac{1}{n}$.
    \item Subjective: An entity assigns probabilities to each possible outcome.
\end{itemize}

Probability theory does not depend on intepretation.

\subsection{Experiments and Events}

Probability allows us to quantify how likely an outcome is to occur. \\

\textbf{Experiments}: Any process in which the possible outcomes can be identified ahead of time.
\textbf{Events}: A well defined set of possible outcomes of the experiment. \\

Although there is controvery in regard to the proper meaning and interpration of some of the
probabilities that are assigned to the outcomes of many experiements, once these probabilities
are assigned, there is complete agreemenet upon the mathematical theory of probability. \\

Almost all work in the mathematical theory of probability is related to:
\begin{itemize}
    \item Methods for determining probabilities of certain events from given probabilities for
    each possible outcome in an experiment.
    \item Methods for revising probabilities of events when additional relevant information is 
    obtained.
\end{itemize}

\subsection{Set Theory}



\end{document}