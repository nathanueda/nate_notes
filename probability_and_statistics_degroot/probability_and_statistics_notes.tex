\documentclass[11pt]{article}
\usepackage{hyperref} 
\usepackage{amsmath, amsfonts, amssymb}
\usepackage{graphicx}
\usepackage{float}
\usepackage[margin=1in]{geometry}

\parindent0px

\emergencystretch=0pt
\pretolerance=150
\tolerance=10000
\hbadness=10000
\hfuzz=0pt

\title{Probability and Statistics Notes}
\author{Nathan Ueda}
\date{\today} 

\begin{document}
\maketitle 
\pagebreak
\tableofcontents 
\pagebreak

\section{Introduction to Probability}
\subsection{The History of Probability}
\subsection{Interpretations of Probability}

Probability Interpretations
\begin{itemize}
    \item Frequency: If an experiment is carried out many times, the frequency with which a
    particular outcome occurred would define its probability.
    \item Classical: If an outcome of some experiment must be one of $n$ different, equally
    likely outcomes, the probability of each outcome is $\frac{1}{n}$.
    \item Subjective: An entity assigns probabilities to each possible outcome.
\end{itemize}

Probability theory does not depend on intepretation.

\subsection{Experiments and Events}

Probability allows us to quantify how likely an outcome is to occur. \\

\textbf{Experiments:} Any process in which the possible outcomes can be identified ahead of time.
\textbf{Events:} A well defined set of possible outcomes of the experiment. \\

Although there is controvery in regard to the proper meaning and interpration of some of the
probabilities that are assigned to the outcomes of many experiements, once these probabilities
are assigned, there is complete agreemenet upon the mathematical theory of probability. \\

Almost all work in the mathematical theory of probability is related to:
\begin{itemize}
    \item Methods for determining probabilities of certain events from given probabilities for
    each possible outcome in an experiment.
    \item Methods for revising probabilities of events when additional relevant information is 
    obtained.
\end{itemize}

\subsection{Set Theory}

\textbf{Sample Space:} The collection of all possible outcomes of an experiment. \\
\textbf{Empty Set:} Subset of $S$ containing no elements, denoted $\emptyset$, representing
any events that cannot occur. \\
\textbf{Complement:} For some set $A$, its complement, denoted $A^c$, is the set containing all 
elements of $S$ not in $A$. \\
\textbf{Union:} For $n$ sets $A_1, \ldots, A_n$, their union, denoted $A_1 \cup \ldots \cup A_n
$ or $ \bigcup_{i=1}^{n} A_i $, is defined as the set containing all outcomes that belong to at
least one of these $n$ sets. \\
\textbf{Intersection:} For $n$ sets $A_1, \ldots, A_n$, their intersection, denoted $A_1 \cap 
\ldots \cap A_n $ or $ \bigcap_{i=1}^{n} A_i $, is defined as the set containing the elements 
common to all these $n$ sets. \\
\textbf{Disjoint/Mutually Exclusive:} Two sets $A$ and $B$ are disjoint/mutually exclusive if 
they have no outcomes in common, that is, if $ A \cap B = \emptyset $, representing that both
$A$ and $B$ cannot occur.

\subsection{The Definition of Probability}

Axioms of Probability:
\begin{enumerate}
    \item For every event $A$, $P(A) \ge 0$
    \item $P(S) = 1$
    \item $P( \bigcup_{i=1}^{\infty} A_i) = \sum_{i=1}^{\infty}P(A_i)$
\end{enumerate}

Basic Theorems:
\begin{enumerate}
    \item $P(\emptyset) = 0$
    \item For every finite sequence of $n$ disjoint events, $A_1, \ldots, A_n$, $P( 
    \bigcup_{i=1}^{n} A_i) = \sum_{i=1}^{n}P(A_i)$
    \item For every event $A$, $P(A^c) = 1 - P(A)$
    \item If $A \subset B$, then $P(A) \le P(B)$
    \item For every event $A$, $0 \le P(A) \le 1$
    \item For every two events $A$ and $B$, $P(A \cap B^c) = P(A) - P(A \cap B)$
    \item For every two events $A$ and $B$, $P(A \cup B) = P(A) + P(B) - P(A \cap B)$
    \item Bonferroni Inequality: For all events $A_1, \ldots, A_n$, $P(\bigcap_{i=1}^{n} A_i) 
    \ge 1 - \sum_{i=1}^{n} P(A_i^c)$
\end{enumerate}

\subsection{Finite Sample Spaces}

Simple Sample Space
\begin{itemize}
    \item Has a finite number ($n$) of possible outcomes
    \item Each outcome has an equal probability ($\frac{1}{n}$)
    \item If an event $A$ has $m$ outcomes, then $P(A) = \frac{m}{n}$
\end{itemize}

\subsection{Counting Methods}

\textbf{Multiplication Rule:} An experiment with $k$ parts where the $i$th part has $n_i$ 
possible outcomes (regardless of which specific outcomes have occurred in the other parts) has 
a sample space $S = n_1 n_2 \ldots n_k$ \\

\textbf{Permutations ($P_{n,k}$):}
\begin{itemize}
    \item Number of ways to arrange a set where order matters
    \item Sampling considering $n$ different items and making $k$ choices from them
    \begin{itemize}
        \item Sampling with replacement: $n^k$
        \item Sampling without replacement: $n(n-1) \ldots (n-k+1)$
        \begin{itemize}
            \item $n$ options for first choice, $n-1$ options for second choice, $n-k+1$ 
            options for $k$th choice
        \end{itemize}
    \end{itemize}
    
    \item The number of permutations of $n$ different items is $P_{n,n} = n$!
    \item The number of permutations of $n$ different items making $k$ choices ($0 \le k \le 
    n$) is
        \[P_{n,k} = n(n-1) \ldots (n-k+1)\]
        \[P_{n,k} = n(n-1) \ldots (n-k+1) \left(\frac{1}{1}\right)\]
        \[P_{n,k} = n(n-1) \ldots (n-k+1) \left(\frac{(n-k)(n-k-1) \ldots 1}{(n-k)(n-k-1) 
        \ldots 1}\right)\]
        \[P_{n,k} = \frac{n(n-1) \ldots (n-k+1)(n-k)(n-k-1) \ldots 1}{(n-k)(n-k-1) \ldots 1}\]
        \[P_{n,k} = \frac{n!}{(n-k)!}\]
\end{itemize}

\subsection{Combinatorial Methods}

\end{document}