\documentclass[11pt]{article}
\usepackage{hyperref} 
\usepackage{amsmath, amsfonts, amssymb}
\usepackage{graphicx}
\usepackage{float}
\usepackage[margin=1in]{geometry}

\parindent0px

\emergencystretch=0pt
\pretolerance=150
\tolerance=10000
\hbadness=10000
\hfuzz=0pt

\title{Python Programming for Economics and Finance Notes}
\author{Nathan Ueda}
\date{\today} 

\begin{document}
\maketitle 
\pagebreak
\tableofcontents 
\pagebreak

\section{Getting Started}

\textbf{Modal Editing:}
There are two modes in Jupyter Notebooks:

\begin{itemize}
    \item Edit Mode: What is typed enters the cell.
    \item Command Mode: Keystrokes are commands (typing \texttt{b} creates a new cell.)
\end{itemize}

To get from Edit Mode from Command Mode: \texttt{Esc} \\
To get from Command Mode from Edit Mode: \texttt{Enter} \\
 
\textbf{Inline Help:}
To get help on a command, execute the command with \texttt{?} at the end. \\
Ex: To get help on \texttt{np.random.randn}, execute \texttt{np.random.randn?}. \\



\end{document}